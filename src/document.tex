\documentclass[a4paper, 10pt]{article}
 
% packages
\usepackage[margin=1in]{geometry} 
\usepackage{amsmath,amsthm,amssymb,tensor,physics}
\usepackage{setspace,multicol,tasks}
\usepackage[shortlabels]{enumitem}
\usepackage{fancyhdr}
\usepackage{graphicx,hyperref}
\usepackage[backend=biber, style=apa]{biblatex}

% set section count.
\setcounter{secnumdepth}{0}

% multi column spacing.
\setlength{\columnseprule}{1pt}

% \doublespacing
\graphicspath{./}

% visual setup
\hypersetup{colorlinks,citecolor=black,filecolor=black,linkcolor=black,urlcolor=black}
\graphicspath{./}

% bib: in case I need outside sources.
% \addbibresource{~~~.bib}

% visual setup
\hypersetup{colorlinks,citecolor=black,filecolor=black,linkcolor=black,urlcolor=black}
\graphicspath{./}

% define math fields
\newcommand{\N}{\mathbb{N}}
\newcommand{\Z}{\mathbb{Z}}
 
% Stolen environments.
\newenvironment{theorem}[2][Theorem]{\begin{trivlist}
\item[\hskip\labelsep{\bfseries #1}\hskip \labelsep {\bfseries #2.}]}{\end{trivlist}}
\newenvironment{lemma}[2][Lemma]{\begin{trivlist}
\item[\hskip\labelsep{\bfseries #1}\hskip \labelsep {\bfseries #2.}]}{\end{trivlist}}
\newenvironment{exercise}[2][Exercise]{\begin{trivlist}
\item[\hskip\labelsep{\bfseries #1}\hskip \labelsep {\bfseries #2.}]}{\end{trivlist}}
\newenvironment{problem}[2][Problem]{\begin{trivlist}
\item[\hskip\labelsep{\bfseries #1}\hskip \labelsep {\bfseries #2.}]}{\end{trivlist}}
\newenvironment{question}[2][Question]{\begin{trivlist}
\item[\hskip\labelsep{\bfseries #1}\hskip \labelsep {\bfseries #2.}]}{\end{trivlist}}
\newenvironment{corollary}[2][Corollary]{\begin{trivlist}
\item[\hskip\labelsep{\bfseries #1}\hskip \labelsep {\bfseries #2.}]}{\end{trivlist}}
\newenvironment{answer}[1][Answer]{\begin{trivlist}
\item[\hskip\labelsep{\textit{#1.}}]}{\end{trivlist}}

% My own defined environments
\newenvironment{case}[2][Case]{\begin{trivlist}
    \item[\hskip\underline{
        \hskip\labelsep{#1}
        \hskip\labelsep{#2.}
    }]}
    {\end{trivlist}}

% header/footer
\pagestyle{fancy}
\fancyhf{}
\fancyhead[L]{mjalen/pde\_strauss\_solutions} % Remember to update.
\fancyhead[R]{Chapter 6: Section 1} 
\fancyfoot[R]{Page \thepage}


% doc start
\begin{document}

\raggedcolumns{}
\tableofcontents

\section{Section 6.1}

\subsection{Question 1}

Show that a function which is a power series in the complex variable $x+iy$ must satisfy the Cauchy-Riemann equations and therefore the Laplace's equation.

\begin{answer}
    Suppose $f(z)$ is an analytic function such that $f(z) = u(x+i y) + i v(x + i y) = \sum_{n=0}^\infty a_n (x + i y)^n$. The functions $u$ and $v$ can be expressed as follows:
    
    \begin{align*}
        u(x+i y) &= - i v(x + i y) + \sum_{n=0}^\infty a_n (x+ i y)^n,\\
        i v(x + i y) &= - u(x + i y) + \sum_{n=0}^\infty a_n (x + i y)^n. 
    \end{align*}

    The partial derivatives of $u$ and $v$ with respect to $x$ and $y$ are

    \begin{align*}
        u_x(x + i y) &= - i v_x (x + i y) + \sum_{n=0}^\infty n \cdot a_n (x + i y)^n,\\
        i u_y(x + i y) &= v_y (x + i y) + i \sum_{n=0}^\infty n \cdot a_n (x + i y)^n.
    \end{align*}

    And,

    \begin{align*}
        i v_x (x + i y) &= - u_x(x+ i y) + \sum_{n=0}^\infty n \cdot a_n (x + i y)^n,\\
        - v_y (x + i y) &= u_y (x + i y) + i \sum_{n=0}^\infty n \cdot a_n (x + i y)^n. 
    \end{align*}

    Plugging these values into the Cauchy-Riemann equations yields the following.

    \begin{align*}
        0   &= \partialderivative{u}{x} - \partialderivative{v}{y},\\
            &= \left[- i v_x (x + i y) + \sum_{n=0}^\infty n \cdot a_n (x + i y)^n \right]
            - \left[ - u_y (x + i y) - i \sum_{n=0}^\infty n \cdot a_n (x + i y)^n  \right],\\
            &= \left[- i v_x (x + i y) + \sum_{n=0}^\infty n \cdot a_n (x + i y)^n \right]
            + \left[ u_y (x + i y) + i \sum_{n=0}^\infty n \cdot a_n (x + i y)^n  \right],\\
            &=   \left[ u_y (x+ iy ) - i v_x(x + iy ) \right] + (1 + i) \sum_{n=0}^\infty n \cdot a_n (x + i y)^n.
    \end{align*}

    And, 

    \begin{align*}
        0   &= \partialderivative{u}{y} + \partialderivative{v}{x},\\
            &= \frac{1}{i} \left[v_y (x + i y) + i \sum_{n=0}^\infty n \cdot a_n (x + i y)^n \right] 
            + \frac{1}{i} \left[ - u_x(x+ i y) + \sum_{n=0}^\infty n \cdot a_n (x + i y)^n \right],\\
            &= \left[ v_y (x + i y) - u_x(x+ i y) \right]
               + (1 + i) \sum_{n=0}^\infty n \cdot a_n ( x + i y)^n 
    \end{align*}

    {\color{blue} Revisit.}
\end{answer} 

\subsection{Question 2}

Find the solutions that depend only on $r$ of the equation $\Delta_3 u = k^2 u$, where $k$ is a positive constant. (\textit{Hint:} Substitute $u = v/r$.)

\begin{answer}
    As provided by the spherical Laplace equation in Equation 6 (p. 159), the PDE becomes the following in terms of $r$.

    \begin{align*}
        u_{r r} + \frac{2}{r} u_r = k^2 u.
    \end{align*}

    Substituting $u = v / r$ and its derivatives derives the following equation. 

    \begin{align*}
        0   &= \left( \frac{v_{r r}}{r} - \frac{2 v_r}{r^2} + \frac{2 v}{ r^3 } \right) 
        + \left( \frac{2}{r} \left[ \frac{v_r}{r} - \frac{v}{r^2}  \right] \right)
        - k^2 \left( \frac{v}{r} \right),\\
            &= \frac{v_{r r}}{r} - k^2 \frac{v}{r},\\
            &= v_{r r} - k^2 v.     
    \end{align*}

    This ODE solves to $v = c_1 \cosh{k r} + c_2 \sinh{k r}$. Thus, the general solution of the provided problem is

    \begin{align*}
        u(r) = \frac{1}{r} \left[ c_1 \cosh{k r} + c_2 \sinh{k r} \right] 
    \end{align*}

    provided $c_1$ and $c_2$ are constant. 
    
    {\color{blue} \textbf{Note:} The solution can also be written in terms of the exponential, $u(r) = \frac{1}{r} \left[ c_1 e^{k r} + c_2 e^{-k r} \right]$.}
\end{answer}

\subsection{Question 3}

Find the solutions that depend only on $r$ of the equation $\Delta_2 u = k^2 u$, where $k$ is a positive constant. (\textit{Hint:} Look up Bessel's differential equation in Section 10.5.)

\begin{answer}
    As provided by the polar Laplace equation in Equation 5 (p. 157), the PDE becomes the following in terms of $r$.

    \begin{align*}
        u_{r r} + \frac{1}{r} u_{r} = k^2 u.
    \end{align*}

    Notice that $-k^2 = (ik)^2$. Then,

    \begin{align*}
        u_{r r} + \frac{1}{r} u_r + (i k)^2 u = 0.
    \end{align*}

    {\color{blue} Revisit. The exercise solution I have writes $u(r)$ in terms of Bessel's function, but I do not buy the derivation they have.}
\end{answer}

\subsection{Question 4}

Solve $\Delta_3 u = 0$ in the spherical shell $0<a<r<b$ with the boundary conditions $u=A$ on $r=a$ and $u=B$ on $r=b$, where $A$ and $B$ are constants. (\textit{Hint:} Look for a solution depending only on $r$.)

\begin{answer}
    The spherical Laplace equation in terms of $r$ is given by
    
    \begin{align*}
        0 = u_{r r} + \frac{2}{r} u_r. 
    \end{align*}

    The solution to this ODE is $u(r) =-c_1 r^{-1} + c_2$. The boundary conditions derive the system of equations below. 
    
    \begin{align*}
        A = u(a) &= - c_1 (a)^{-1} + c_2,\\
        B = u(b) &= - c_1 (b)^{-1} + c_2.
    \end{align*}

    Subtracting these equations derives $c_1$.
    
    \begin{align*}
        A - B &= b^{-1} c_1 - a^{-1} c_1,\\
              &= \left( \frac{1}{b} - \frac{1}{a} \right) c_1,\\
              &= \frac{a - b}{a b} c_1,\\
        \frac{(A - B)a b}{a - b}  &= c_1.
    \end{align*}

    Substitution into the first equation yields $c_2$. 

    \begin{align*}
        A &= - \frac{(A - B)a b}{a - b} \cdot (a)^{-1} + c_2,\\
          &= - \frac{(A - B) b}{a - b} + c_2,\\
        A + \frac{(A - B) b}{a - b} &= c_2.
    \end{align*}

    Thus, the solution is 

    \begin{align*}
        u(r) &= - \left[\frac{(A - B)a b}{a - b} \right] r^{-1} + A + \frac{(A - B) b}{a - b},\\
             &= A + \frac{(A - B) b}{a - b} \cdot \left[1 - a r^{-1} \right].
    \end{align*}
\end{answer}

\end{document}
