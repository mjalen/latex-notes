\documentclass[a4paper, 10pt]{article}
 
% packages
\usepackage[margin=1in]{geometry} 
\usepackage{amsmath,amsthm,amssymb,tensor,physics}
\usepackage{setspace,multicol,tasks}
\usepackage[shortlabels]{enumitem}
\usepackage{fancyhdr}
\usepackage{graphicx,hyperref}
\usepackage[backend=biber, style=apa]{biblatex}

% set section count.
\setcounter{secnumdepth}{0}

% multi column spacing.
\setlength{\columnseprule}{1pt}

% \doublespacing
\graphicspath{./}

% visual setup
\hypersetup{colorlinks,citecolor=black,filecolor=black,linkcolor=black,urlcolor=black}
\graphicspath{./}

% bib: in case I need outside sources.
% \addbibresource{~~~.bib}

% visual setup
\hypersetup{colorlinks,citecolor=black,filecolor=black,linkcolor=black,urlcolor=black}
\graphicspath{./}

% define math fields
\newcommand{\N}{\mathbb{N}}
\newcommand{\Z}{\mathbb{Z}}
 
% Stolen environments.
\newenvironment{theorem}[2][Theorem]{\begin{trivlist}
\item[\hskip\labelsep{\bfseries #1}\hskip \labelsep {\bfseries #2.}]}{\end{trivlist}}
\newenvironment{lemma}[2][Lemma]{\begin{trivlist}
\item[\hskip\labelsep{\bfseries #1}\hskip \labelsep {\bfseries #2.}]}{\end{trivlist}}
\newenvironment{exercise}[2][Exercise]{\begin{trivlist}
\item[\hskip\labelsep{\bfseries #1}\hskip \labelsep {\bfseries #2.}]}{\end{trivlist}}
\newenvironment{problem}[2][Problem]{\begin{trivlist}
\item[\hskip\labelsep{\bfseries #1}\hskip \labelsep {\bfseries #2.}]}{\end{trivlist}}
\newenvironment{question}[2][Question]{\begin{trivlist}
\item[\hskip\labelsep{\bfseries #1}\hskip \labelsep {\bfseries #2.}]}{\end{trivlist}}
\newenvironment{corollary}[2][Corollary]{\begin{trivlist}
\item[\hskip\labelsep{\bfseries #1}\hskip \labelsep {\bfseries #2.}]}{\end{trivlist}}
\newenvironment{answer}[1][Answer]{\begin{trivlist}
\item[\hskip\labelsep{\textit{#1.}}]}{\end{trivlist}}

% My own defined environments
\newenvironment{case}[2][Case]{\begin{trivlist}
    \item[\hskip\underline{
        \hskip\labelsep{#1}
        \hskip\labelsep{#2.}
    }]}
    {\end{trivlist}}

% header/footer
% TODO Remember to update.
\pagestyle{fancy}
\fancyhf{}
\fancyhead[L]{Jalen Moore} 
\fancyhead[R]{Homework 1: MATH 4200} 
\fancyfoot[R]{Page \thepage}


% doc start
\begin{document}

\raggedcolumns{}
\tableofcontents

\section{Section 1.1}

% 2A, 5A, 11, 18

\subsection{Exercise 2A}

Show that the following equation has at least one solution in the given interval.

\begin{align*}
	\sqrt{x} - \cos{x} = 0, \left[ 0,1 \right].
\end{align*}

\begin{answer}
	Notice that $\sqrt{x}$ and $\cos{x}$ are both continuous for $[0,1]$. Therefore, let $f\in C[0,1]$ such that $f(x) = \sqrt{x} - \cos{x}$. Then,

	\begin{align*}
		f(0) &= \sqrt{0} - \cos{0} = 0 - 1 = -1,\\
		f(1) &= \sqrt{1} - \cos{1} = 1 - 0.5403 = 0.4597.  
	\end{align*}
\end{answer}

Thus by the Intermediate Value Theorem, there exists a root $c\in (0,1)$ such that $f(c)=0$.

\subsection{Exercise 5A}

Find $\max_{a\leq x \leq b} | f(x) |$ for the following function and interval.

\begin{align*}
	f(x) = \frac{(2 - e^x + 2x)}{3}, [0,1]
\end{align*}

\begin{answer}
	The function $f(x)$ is given such that $f\in C[0,1]$. The derivative of $f(x)$ is $f'(x) = \frac{- e^x + 2}{3}$. Solving this derivative derives the following. 

	\begin{align*}
		0 &= \frac{- e^x + 2}{3},\\
		0 &= - e^x + 2,\\
		e^x &= 2,\\
		x &= \ln{2}.
	\end{align*}

	Therefore by the Extreme Value Theorem, the extrema must occur at $x\in \{ 0, \ln{2}, 1 \}$. Therefore,

	\begin{align*}
		f(0) &= \frac{[2 - e^0 + 2\cdot (0)]}{3} = \frac{1}{3},\\
		f(\ln{2}) &= \frac{(2 - e^{\ln{2}} + 2\ln{2})}{3} \approx - 0.4621,\\ 
		f(1) &= \frac{(2 - e^1 + 2\cdot (1))}{3} = \frac{4 -e^1}{3} \approx - 0.4272.
	\end{align*}

	Thus, the absolute value of these possible extrema provides 

	\begin{align*}
		\max_{x\in [0,1]} \left| \frac{(2 - e^x + 2x)}{3} \right| = 0.4621.
	\end{align*}
\end{answer}

\subsection{Exercise 11}

Find the second Taylor polynomial $P_2(x)$ for the function $f(x)=e^x \cos{x}$ about $x_0 = 0$.

\begin{enumerate}[(a)]
	\item Use $P_2(0.5)$ to approximate $f(0.5)$. Find an upper bound for error $\left| f(0.5) - P_2(0.5) \right|$ using the error formula and compare it to the actual error.
	\item Find a bound for the error $\left| f(x) - P_2(x) \right|$ in using $P_2(x)$ to approximate $f(x)$ on the interval $[0,1]$.
	\item Approximate $\int_0^1 f(x) \dd{x}$ using $\int_0^1 P_2(x) \dd{x}$.
	\item Find an upper bound for the error in (c) using $\int_0^1 \left| R_2(x) \dd{x} \right|$ and compare the bound to the actual error.
\end{enumerate}

\begin{answer}
	The second Taylor polynomial for $f(x)$ is

	{\color{blue} Organize this, this is messy.}

	\begin{align*}
		P_2(x) &= \sum_{k=0}^2 \frac{f^{(k)}(x_0)}{k!} (x - x_0)^k,\\
			   &= e^{x_0} \cos{x_0}\\
			   &+ e^{x_0} \left[ \cos{x_0} - \sin{x_0} \right] x\\
			   &+ \frac{e^{x_0} \left[\cos{x_0} - \sin{x_0} - \sin{x_0} - \cos{x_0} \right]}{2} x^2,\\
			   &= e^{x_0} \cos{x_0}\\
			   &+ e^{x_0} \left[ \cos{x_0} - \sin{x_0} \right] x\\
			   &- e^{x_0} \sin{x_0} x^2,\\
			   &= 1 + \left[ 1 - 0  \right] x + 0 \cdot x^2,\\
			   &= 1 + x.
	\end{align*}

	Therefore, 

	\begin{enumerate}[(a)]
		\item The actual and approximate values are 

			\begin{align*}
				f(0.5) &= e^{0.5} \cos{(0.5)} = 1.4468890366,\\
				P_2(0.5) &= 1 + 0.5 = 1.5.
			\end{align*}

			To find the upper bound error, the third derivative of $f$ is

			\begin{align*}
				f^{(3)}(x) &= -2 e^{x} \left[ \sin{x} + \cos{x} \right]. 
			\end{align*}

			Therefore, the upper bound error is

			\begin{align*}
				| f(0.5) - P_2(0.5) | &= | R_2 (0.5) |,\\
				&\leq \frac{f^{(3)}(\xi)}{3!} (0.5 - 0)^{3},\\
				&= \frac{-2 e^\xi \left[ \sin\xi + \cos\xi \right]}{6} (0.5)^3,\\
				&\leq \max_\xi \frac{-2e^\xi \left[ \sin\xi + \cos\xi \right]}{6} \cdot 0.125,\\
				&= \frac{-2 e^{0.5} \left[ \sin{0.5} + \cos{0.5} \right]}{6} \cdot 0.125,\\ 
				&= 0.0932.
			\end{align*}

			{\color{blue} Make sure the absolute value removal makes sense (Hint: They don't.}

			The actual error is

			\begin{align*}
				| f(0.5) - P_2(0.5) | &= | 1.446889 - 1.5 |,\\
				&= | - 0.053111 |,\\
				&= 0.053111.
			\end{align*}

		\item The non-fixed upper bound error for $n=2$ is the following.

			\begin{align*}
				| f(x) - P_2(x) | &= | R_2 (x) |,\\
				&= | \frac{f^{(3)}(\xi)}{3!} x^3 |,\\
				&\leq \max_{\xi \in [0,1]} \max_{x\in [0, 1]} \left| \frac{f^{(3)}(\xi)}{3!} x^3 \right|,\\
				&= \max_{\xi \in [0,1]} \max_{x\in [0, 1]} \left| \frac{-2 e^\xi \left[ \sin{\xi} + \cos{\xi} \right]}{6} x^3 \right|,\\ 
				&= \max_{\xi \in [0,1]} \left| \frac{-2 e^\xi \left[ \sin{\xi} + \cos{\xi} \right]}{6} \right|,\\
				&= 1.2520.
			\end{align*}

			{\color{blue} Perhaps add more derivation for max.}

		\item 

			\begin{align*}
				\int_0^1 f(x) \dd{x} &\approx \int_0^1 P_2(x) \dd{x},\\
				&= \int_0^1 1 + x \dd{x},\\
				&= x + \frac{x^2}{2} \Bigm|_0^1,\\
				&= \left( 1 + \frac{1}{2} \right) - 0,\\
				&= 1.5.
			\end{align*}

		\item The upper bound error is 

			\begin{align*}
				\left| \int_0^1 f(x) \dd{x} - \int_0^1 P_2(x) \dd{x} \right| 
				&= \left| \int_0^1 R_2(x) \dd{x} \right|,\\
				&= \left| \int_0^1 \frac{-2 e^\xi \left[ \sin{\xi} + \cos{\xi} \right]}{6} x^3 \dd{x} \right|,\\
				&= \left| \frac{-2 e^\xi \left[ \sin{\xi} + \cos{\xi} \right]}{24} x^4 \Bigm|_0^1  \right|,\\
				&= \left| \frac{-2 e^\xi \left[ \sin{\xi} + \cos{\xi} \right]}{24} \right|,\\
				&\leq \max_{\xi\in[0,1]} \left| \frac{-2 e^\xi \left[ \sin{\xi} + \cos{\xi} \right]}{24} \right|,\\
				&= 0.313.
			\end{align*}

			In contrast, the actual error is

			\begin{align*}
				\left| \int_0^1 f(x) \dd{x} - \int_0^1 P_2(x) \dd{x} \right| 
				&= \left| \int_0^1 e^x \cos{x} \dd{x} - 1.5 \right|,\\
				&= \left| \frac{e^x \cos{x} + e^x \sin{x}}{2} \Bigm|_0^1 - 1.5 \right|,\\
				&= \left| \frac{e^1 \cos{1} + e^1 \sin{1}}{2} - \frac{1}{2} - 1.5 \right|,\\
				&= \left| 1.8780 - 2 \right|,\\
				&= 0.1220.
			\end{align*}
	\end{enumerate}

	{\color{blue} Clean up this problem. The answers are right, but it is messy.}
\end{answer}

\subsection{Exercise 18}

Let $f(x)=(1-x)^{-1}$ and $x_0=0$. Find the $n$th Taylor polynomial $P_n(x)$ for $f(x)$ about $x_0$. Find a value of $n$ necessary for $P_n(x)$ to approximate $f(x)$ to within $10^{-6}$ on $[0, 0.5]$.

\begin{answer}
	Explicitly deriving the first few derivatives of $f(x)$ yields:

	\begin{align*}
		f^{(0)}(x) &= (1-x)^{-1} = 0! (1-x)^{-1},\\
		f^{(1)}(x) &= (1-x)^{-2} = 1! (1-x)^{-2},\\
		f^{(2)}(x) &= 2(1-x)^{-3} = 2! (1-x)^{-3},\\
		f^{(3)}(x) &= 6(1-x)^{-4} = 3! (1-x)^{-4},\\
		f^{(4)}(x) &= 24(1-x)^{-5} = 4! (1-x)^{-5}\ldots
	\end{align*}

	Therefore, from this pattern the formula for the $k$th derivative must be 

	\begin{align*}
		f^{(k)}(x) &= k! (1-x)^{-(k+1)}.
	\end{align*}

	Therefore, the $n$th Taylor polynomial is

	\begin{align*}
		P_n(x; x_0 = 0) &= \sum_{k=0}^n \frac{f^{(k)}(x_0)}{k!} (x - x_0)^k,\\
		&= \sum_{k=0}^n (1-x_0)^{-(k+1)} (x - x_0)^k,\\
		&= \sum_{k=0}^n 1^{-(k+1)} x^k,\\
		&= \sum_{k=0}^n x^k.
	\end{align*}

	The remainder is

	\begin{align*}
		R_n (x) &= \frac{f^{(n+1)}(\xi)}{(n+1)!} x^{n+1},\\
				&= (1-\xi)^{-(n+2)} x^{n+1}.
	\end{align*}

	To approximate $f(x)$ within $10^{-6}$, the following must be true

	\begin{align*}
		10^{-6} &\geq \left| f(x) - P_n (x) \right|,\\
				&= \left| R_n (x) \right|,\\
				&= \left| (1-\xi)^{-(n+2)} x^{n+1} \right|.
	\end{align*}

	The upper bound error is then

	\begin{align*}
		10^{-6} &\geq \max_{x\in [0,0.5]} \max_{\xi\in [0,0.5]} \left| (1-\xi)^{-(n+2)} x^{n+1} \right|,\\
				&= \max_{x\in [0,0.5]}
	\end{align*}
\end{answer}

\section{Section 1.2}

% 14, 15A, 21, 22, 23A, 28

\section{Section 1.3}

% 6B, 7C, 8, 9

\end{document}
